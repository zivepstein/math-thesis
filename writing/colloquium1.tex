	\documentclass[12pt]{article}
\usepackage{color}

% This first part of the file is called the PREAMBLE. It includes
% customizations and command definitions. The preamble is everything
% between \documentclass and \begin{document}.

\usepackage[margin=1in]{geometry}  % set the margins to 1in on all sides
\usepackage{graphicx}              % to include figures
\usepackage{amsmath}               % great math stuff
\usepackage{amsfonts}              % for blackboard bold, etc
\usepackage{amsthm}                % better theorem environments
\usepackage{changepage}
\usepackage{lipsum}                     % Dummytext
\usepackage{xargs}  
\usepackage{amssymb}                    % Use more than one optional parameter in a new commands
\usepackage[pdftex,dvipsnames]{xcolor}  % Coloured text etc.
% 
\usepackage[colorinlistoftodos,prependcaption,textsize=tiny]{todonotes}
\newcommandx{\unsure}[2][1=]{\todo[linecolor=red,backgroundcolor=red!25,bordercolor=red,#1]{#2}}
\newcommandx{\change}[2][1=]{\todo[linecolor=blue,backgroundcolor=blue!25,bordercolor=blue,#1]{#2}}
\newcommandx{\info}[2][1=]{\todo[linecolor=OliveGreen,backgroundcolor=OliveGreen!25,bordercolor=OliveGreen,#1]{#2}}
\newcommandx{\improvement}[2][1=]{\todo[linecolor=Plum,backgroundcolor=Plum!25,bordercolor=Plum,#1]{#2}}
\newcommandx{\thiswillnotshow}[2][1=]{\todo[disable,#1]{#2}}

% various theorems, numbered by section

\newtheorem{thm}{Theorem}[section]
\newtheorem{lem}[thm]{Lemma}
\newtheorem{prop}[thm]{Proposition}
\newtheorem{cor}[thm]{Corollary}
\newtheorem{conj}[thm]{Conjecture}

\DeclareMathOperator{\id}{id}

\newcommand{\bd}[1]{\mathbf{#1}}  % for bolding symbols
\newcommand{\RR}{\mathbb{R}}      % for Real numbers
\newcommand{\ZZ}{\mathbb{Z}}      % for Integers
\newcommand{\col}[1]{\left[\begin{matrix} #1 \end{matrix} \right]}
\newcommand{\comb}[2]{\binom{#1^2 + #2^2}{#1+#2}}
\usepackage{graphicx}
\usepackage{csquotes}
\usepackage{lipsum}
\newcommand\tab[1][1cm]{\hspace*{#1}}

\begin{document}


\nocite{*}

\title{Colloquium Attendance \#1: Calculus and Prime Numbers \\ By Shahriar Shahriari}


\author{Ziv Epstein \\ 
	\texttt{ziv.epstein@pomona.edu}}

\maketitle

%Attend at least two colloquiums—Wednesdays at 4:15 at CMC, cookies at 3:45—or research seminars this semester and write a paragraph about each talk. Your paragraph can be about the topic of the colloquium or the strengths and weaknesses of the presentation style. We urge you to go to colloquiums early in the semester, since you may decide that you want to go to more than the required two! Ideally, you will submit your paragraph no later than two days after the talk. However, we will accept them up to ten days after the talk.
\textbf{Topic}: Calculus is about continuous, smooth, differential functions, but functions describing primes typically are not. The talk explores the extent to which calculus can be used to the study of primes. It beings by defining fermat numbers $F_n=2^{2^n}+1$,which are conjectured to be prime. unfortunately, for $n>4$, not prime!

Then, some open questions were explored:
\begin{enumerate}
	\item Are any $F_n$ prime?
	\item  Are the number of $F_n$ prime finite? 
	\item Is $F_{33}$ prime or composite?
\end{enumerate}
Then, the Mersenne numbers, $M_p = 2^p-1$ where $p$ is prime, were defined. Some interesting facts were then explored. For example, the largest prime known is $M_{74,207,281}$! Then the notion of prime deserts where discussed. A proof for why prime deserts of arbitrary length, say $k$, necessarily exist. This is because $k! + 1,\cdots ,k$ is necessarily a $k$ prime desert.

The talk then pivoted to discuss the search for a smooth function to approximate $\pi(x)$, which results in the prime number theorem, which states: $$\lim_{x \to \infty} \frac{\pi(x)}{x/\ln(x)} = 1$$
In this vein, the talk explored an experiment by Guass, where he counted $6272$ primes between 2,600,000 and 2,700,000. He then computed $$\int_{2,600,000}^{2,600,000}\frac{dt}{\ln t} = 6271.72$$ by hand. Very surprising how similar they are!
The talk concluded by discussing applications to the Riemann hypothesis. 

\textbf{Strengths}: Introduces topics well with examples and relating back to course material. Very clear, interesting and exciting!

\textbf{Weaknesses:} $\varnothing$

\end{document}