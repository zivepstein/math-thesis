\documentclass[12pt]{article}
\usepackage[]{algorithm2e}
\usepackage{tikz}
\usepackage{ifthen}
% This first part of the file is called the PREAMBLE. It includes
% customizations and command definitions. The preamble is everything
% between \documentclass and \begin{document}.
\usepackage{amsthm}
\newtheorem{theorem}{Theorem}
\usepackage[margin=1in]{geometry}  % set the margins to 1in on all sides
\usepackage{graphicx}              % to include figures
\usepackage{amsmath}               % great math stuff
\usepackage{amsfonts}              % for blackboard bold, etc
\usepackage{amsthm}                % better theorem environments
\usepackage{changepage}
\usepackage{lipsum}                     % Dummytext
\usepackage{xargs}                      % Use more than one optional parameter in a new commands
 % Coloured text etc.
% 
\usepackage[colorinlistoftodos,prependcaption,textsize=tiny]{todonotes}
\newcommandx{\unsure}[2][1=]{\todo[linecolor=red,backgroundcolor=red!25,bordercolor=red,#1]{#2}}
\newcommandx{\change}[2][1=]{\todo[linecolor=blue,backgroundcolor=blue!25,bordercolor=blue,#1]{#2}}
\newcommandx{\info}[2][1=]{\todo[linecolor=OliveGreen,backgroundcolor=OliveGreen!25,bordercolor=OliveGreen,#1]{#2}}
\newcommandx{\improvement}[2][1=]{\todo[linecolor=Plum,backgroundcolor=Plum!25,bordercolor=Plum,#1]{#2}}
\newcommandx{\thiswillnotshow}[2][1=]{\todo[disable,#1]{#2}}
\newcommand{\blockmatrix}[9]{
	\draw[draw=#4,fill=#5] (0,0) rectangle( #1,#2);
	\ifthenelse{\equal{#6}{true}}
	{
		\draw[draw=#7,fill=#8] (0,#2) -- (#9,#2) -- ( #1,#9) -- ( #1,0) -- ( #1 - #9,0) -- (0,#2 -#9) -- cycle;
	}
	{}
	\draw ( #1/2, #2/2) node { #3};
}

% Quick implementation of a tikz right parenthesis
% \rightparen{width}
\newcommand{\rightparen}[1]{
	\begin{tikzpicture} 
	\draw (0,#1/2) arc (0:30:#1);
	\draw (0,#1/2) arc (0:-30:#1);
	\end{tikzpicture}%this comment is necessary
}

% Quick implementation of a tikz left parenthesis
% \leftparen{width}
\newcommand{\leftparen}[1]{
	\begin{tikzpicture} 
	\draw (0,#1/2) arc (180:150:#1);
	\draw (0,#1/2) arc (180:210:#1);
	\end{tikzpicture}%this comment is necessary
}

% Unframed block matrix, "m" prefix to match fbox, mbox
% \blockmatrix[r,g,b]{width}{height}{text}
\newcommand{\mblockmatrix}[4][none]{
	\begin{tikzpicture} 
	\ifthenelse{\equal{#1}{none}}
	{
		\blockmatrix{#2}{#3}{#4}{none}{none}{false}{none}{none}{0.0}
	}
	{
		\definecolor{fillcolor}{rgb}{#1}
		\blockmatrix{#2}{#3}{#4}{none}{fillcolor}{false}{none}{none}{0.0}
	}
	\end{tikzpicture}%this comment is necessary
}

% Framed block matrix
% \fblockmatrix[r,g,b]{width}{height}{text}
\newcommand{\fblockmatrix}[4][none]{
	\begin{tikzpicture} 
	\ifthenelse{\equal{#1}{none}}
	{
		\blockmatrix{#2}{#3}{#4}{black}{none}{false}{none}{none}{0.0}
	}
	{
		\definecolor{fillcolor}{rgb}{#1}
		\blockmatrix{#2}{#3}{#4}{black}{fillcolor}{false}{none}{none}{0.0}
	}
	\end{tikzpicture}%this comment is necessary
}

% Diagonal block matrix
% \dblockmatrix[r,g,b]{width}{height}{text}
\newcommand{\dblockmatrix}[4][none]{
	\begin{tikzpicture} 
	\ifthenelse{\equal{#1}{none}}
	{
		\blockmatrix{#2}{#3}{#4}{black}{none}{true}{black}{none}{0.35cm}
	}
	{
		\definecolor{fillcolor}{rgb}{#1}
		\blockmatrix{#2}{#3}{#4}{black}{none}{true}{black}{fillcolor}{0.35cm}
	}
	\end{tikzpicture}%this comment is necessary
}


% Diagonal block matrix, but exposes diagonal offset
% \diagonalblockmatrix[r,g,b]{width}{height}{text}
\newcommand{\diagonalblockmatrix}[5][none]{
	\begin{tikzpicture} 
	
	\ifthenelse{\equal{#1}{none}}
	{
		\blockmatrix{#2}{#3}{#4}{black}{none}{true}{black}{none}{#5}
	}
	{
		\definecolor{fillcolor}{rgb}{#1}
		\blockmatrix{#2}{#3}{#4}{black}{none}{true}{black}{fillcolor}{#5}
	}
	
	\end{tikzpicture}%necessary comment
}

\newcommand{\valignbox}[1]{
	\vtop{\null\hbox{#1}}% necessary comment
}

% a hack so that I don't have to worry about the number of columns or
% spaces between columns in the tabular environment
\newenvironment{blockmatrixtabular}
{% necessary comment
	\begin{tabular}{
			@{}l@{}l@{}l@{}l@{}l@{}l@{}l@{}l@{}l@{}l@{}l@{}l@{}l@{}l@{}l@{}l@{}l@{}l@{}l
			@{}l@{}l@{}l@{}l@{}l@{}l@{}l@{}l@{}l@{}l@{}l@{}l@{}l@{}l@{}l@{}l@{}l@{}l@{}l
			@{}l@{}l@{}l@{}l@{}l@{}l@{}l@{}l@{}l@{}l@{}l@{}l@{}l@{}l@{}l@{}l@{}l@{}l@{}l
			@{}
		}
	}
	{
	\end{tabular}%necessary comment
}
% various theorems, numbered by section

\newtheorem{thm}{Theorem}[section]
\newtheorem{lem}[thm]{Lemma}
\newtheorem{prop}[thm]{Proposition}
\newtheorem{cor}[thm]{Corollary}
\newtheorem{conj}[thm]{Conjecture}

\DeclareMathOperator{\id}{id}

\newcommand{\bd}[1]{\mathbf{#1}}  % for bolding symbols
\newcommand{\RR}{\mathbb{R}}      % for Real numbers
\newcommand{\ZZ}{\mathbb{Z}}      % for Integers
\newcommand{\col}[1]{\left[\begin{matrix} #1 \end{matrix} \right]}
\newcommand{\comb}[2]{\binom{#1^2 + #2^2}{#1+#2}}
\usepackage{graphicx}
\usepackage{csquotes}
\usepackage{lipsum}
\newcommand\tab[1][1cm]{\hspace*{#1}}

\begin{document}


\nocite{*}

\title{Hierarchical Topic Modelling \\ 
	Mathematics Senior Thesis}


\author{Ziv Epstein \\ 
	\texttt{ziv.epstein@pomona.edu}}

\maketitle
\newpage
\tableofcontents
\newpage

\section{Introduction to Topic Modeling}
Here we introduce the notion of topic modeling, cite previous work \cite{lee1999learning} \cite{ho2008nonnegative} and talk about its relation to society. As you can see from first draft, this is mostly done already.

\section{Non Negative Matrix Factorization}
\subsection{How NNMF solves Topic Modeling}
Talk about the linear algrebraic intuition behind NNMF and how it solves topic modeling. This is also already done. 
\subsection{Hierarchical Topic Modellng }
Introduce the core idea of the thesis, which is extending the ideas of NMF to the hierarchical domain. Give some examples, in the vein of \cite{griffiths2004hierarchical}
\section{Algorithms}
We then discuss methods for using NNMF to generate a hierarchical topic model.
\subsection{Single Linkage Graph Construction}
This is the algorithm we invited. My first draft already fully explores this. 

\subsection{Semi-Supervised NNMF}
Here, I will fully describe the ideas and implementation details of SSNMF. I will talk about the intuition behind it and how it introduces notions of SVMs into NNMF.

\subsection{Deep Semi NMF}
Here I will give a brief overview of the notion of neural networks as a generalization of nested matrix factorization. I will describe backpropogation as an optimization scheme and compare it with other methods. I will this describe two works that use look at deep NMF, one of which is motivated by these neural networks, the other of which is motivated by more traditional techniques.
\section{Visualization}
Here I will motivate the visualization of this data, provide some heuristics, survey the field of hierarchical data visualization and propose my alterations. 

\section{Hierarchical Topic Model}
We will then construct our own model for learning a hierarchy of topics within the model itself. This will be a blend of the Single Linkage graph construction model and the Deep Semi NMF.

\section{Results}
We will then apply the method of our Hierarchical Topic Model to several datasets, and compare our results with previous work.
\subsection{Synthetic Data}
We will generate synthetic data with hierarchical topics and verify that our algorithm succesfully extracts them 
\subsection{Standard Data}
We will run our algorithm on data cannonically associated with this task. For our purposes, the 20 News Group Data set will probably be sufficient.
\subsection{Afghan Data}
A collaborator at NYU Abu-Dhabi has hand-curated a dataset of Afghani magazines, which is notable for sociologists. We will run our algorithm on this data set and see what happens

\section{Discussion/Conclusion}
Here we will conclude by situating this work within the field, compare its results with similar methods and propose future work.
\bibliographystyle{plain}

\bibliography{myBib}


\end{document}