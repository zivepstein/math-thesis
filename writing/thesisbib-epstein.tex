	\documentclass[12pt]{article}
\usepackage{color}

% This first part of the file is called the PREAMBLE. It includes
% customizations and command definitions. The preamble is everything
% between \documentclass and \begin{document}.

\usepackage[margin=1in]{geometry}  % set the margins to 1in on all sides
\usepackage{graphicx}              % to include figures
\usepackage{amsmath}               % great math stuff
\usepackage{amsfonts}              % for blackboard bold, etc
\usepackage{amsthm}                % better theorem environments
\usepackage{changepage}
\usepackage{lipsum}                     % Dummytext
\usepackage{xargs}                      % Use more than one optional parameter in a new commands
\usepackage[pdftex,dvipsnames]{xcolor}  % Coloured text etc.
% 
\usepackage[colorinlistoftodos,prependcaption,textsize=tiny]{todonotes}
\newcommandx{\unsure}[2][1=]{\todo[linecolor=red,backgroundcolor=red!25,bordercolor=red,#1]{#2}}
\newcommandx{\change}[2][1=]{\todo[linecolor=blue,backgroundcolor=blue!25,bordercolor=blue,#1]{#2}}
\newcommandx{\info}[2][1=]{\todo[linecolor=OliveGreen,backgroundcolor=OliveGreen!25,bordercolor=OliveGreen,#1]{#2}}
\newcommandx{\improvement}[2][1=]{\todo[linecolor=Plum,backgroundcolor=Plum!25,bordercolor=Plum,#1]{#2}}
\newcommandx{\thiswillnotshow}[2][1=]{\todo[disable,#1]{#2}}

% various theorems, numbered by section

\newtheorem{thm}{Theorem}[section]
\newtheorem{lem}[thm]{Lemma}
\newtheorem{prop}[thm]{Proposition}
\newtheorem{cor}[thm]{Corollary}
\newtheorem{conj}[thm]{Conjecture}

\DeclareMathOperator{\id}{id}

\newcommand{\bd}[1]{\mathbf{#1}}  % for bolding symbols
\newcommand{\RR}{\mathbb{R}}      % for Real numbers
\newcommand{\ZZ}{\mathbb{Z}}      % for Integers
\newcommand{\col}[1]{\left[\begin{matrix} #1 \end{matrix} \right]}
\newcommand{\comb}[2]{\binom{#1^2 + #2^2}{#1+#2}}
\usepackage{graphicx}
\usepackage{csquotes}
\usepackage{lipsum}
\newcommand\tab[1][1cm]{\hspace*{#1}}

\begin{document}


\nocite{*}

\title{Annotated Bibliography for Senior Thesis}


\author{Ziv Epstein \\ 
	\texttt{ziv.epstein@pomona.edu}}

\maketitle
%patterns. human. Drone logic creates a new form of divide between those with access to network and those two dont. and those allow for access of knowledge. different ways of thinking about information and causality.  ``too big to know." knowledge as actional information is post narrative, post casual and post exlanatory. How do you respond? return to narrative. 


\section{The sources}
We begin with the two foundation papers introducing the NNMF concept and standard algorithms. Lee and Seung \cite{lee1999learning} were the first to introduce this idea, and to propose that topic modelling could be thought of as a matrix factorization problem. Ho \cite{ho2008nonnegative} expands on this notion by elaborating on optimization schemas and corresponding algorithms that we will be taking advantage of. We then consider the work of Griffiths and Tenenbaum \cite{griffiths2004hierarchical}, who extened the notion of topic models to a hierarchical domain. We aim to replicate this structure but using an NNMF implementation instead of Latent Dirlichet Allocation (LDA). 

Our method to do is construct a distance matrix from the topic repersentations generated from NNMF. With this graph, we will employ community detection algorithms \cite{fortunato2010community} to build the hierarchy. In particular, there already exist methods for generating a hierarchical community structure within complex networks \cite{lancichinetti2009detecting} that we will take advange of to build our topic model.

\bibliographystyle{plain}

\bibliography{myBib}


\end{document}